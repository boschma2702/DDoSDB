\section{Proposal} \label{sec:introduction}
A Denial of Service (DoS) attack is an attack that aims to disable services of a target system. There are two main types of DoS attacks: \textit{vulnerability DoS} and \textit{flood DoS} \cite{Lin2013}. In one hand, a vulnerability DoS aims to exploit a vulnerability of a target system to reduce its performance or render it useless. An example of such an attack is to send a malformed message to the target machine which can not deal with this message and as a result stops working. On the other hand, a flood DoS attack tries to exhaust the resources of the target. An example of such an attack is to fill the entire bandwidth of the target with messages of the attacker. The attacker can accomplish such bandwidth flood by using multiple machines to produce traffic. When multiple machines are used in the attack, it is called a Distributed Denial of Service (DDoS) attack.  



%An explanation for this increase in frequency could be the rise of \textit{booters} \cite{Santana2017}. A booter, also called stressers, are websites that offer DoS attacks via a website. Booters eliminate the need for any technical knowledge to launch an attack. Clients of booters only need to pay a couple of dollars to launch an attack.

DDoS attacks have increased in power and frequency. In 2011, the peak attack was measured at 60 Gb/s \cite{Arbor2014}, in 2015, 500 Gb/s and in 2016 1.1 Tb/s \cite{Akamai2017}. In the third quartile of 2016 more than 5000 attacks where observed, whereas 200 in the entire 2012 \cite{Akamai2016}. As the number of attacks increase and downtime costs are exceeding on average \$300K per hour \cite{ITIC2016} a need for an efficient and effective mitigation method has become crucial. The first task before the mitigation is the detection of an attack. Intrusion Detection Systems (IDS) are such systems that can fulfill this task. An IDS is a system that monitors a system or network for malicious and/or suspicious activities. 
%Two main categories of IDSs are present: \textit{host-based} and \textit{network-based} \cite{Fallahi2016}. A Host-based IDS (HIDS) is only for a single machine. A Network-based IDS (NIDS) is for an entire network. A NIDS usually scans the incoming and outgoing traffic of a network endpoint. 
Based on the detection methods of IDSs, two categories can be identified: \textit{Anomaly-based} and \textit{Signature-based} \cite{fragkiadakis2015anomaly}. An Anomaly-based IDS (AIDS) bases its detection on a constructed baseline and detects deviations from this baseline. A Signature-based IDS (SIDS) bases its detection on key characteristic of an attack for which predefined signatures are known. An AIDS has as benefit that it can detect unknown attacks but with the weaknesses that it has a low accuracy, needs time to learn a baseline of a system and has difficulties to trigger alerts before an attack scales up. A SIDS has as benefit that it has a high accuracy but with the weaknesses that it is ineffective in detecting unknown attacks and it is hard to maintain an up to date signature list \cite{Liao2013}. 

Our hypothesis is that due to the high accuracy a SIDS is a suitable system that can fulfill the requirement of successfully and efficiently detecting DDoS attacks when the major downside of keeping an up to date signature list is tackled. The solution for this problem is to generate signatures for new attacks. This can be done either manually or automatically. As a manual approach requires significant amount of manual effort \cite{Lin2013}, we propose an automatic method. For this research we generate signatures from extracted features of DDoS attacks for the Bro SIDS\footnote{\url{https://www.bro.org/}}. Bro is an open source network security monitor that offers the functionality of a SIDS. The features of DDoS attacks are extracted by a different research of DDoSDB\footnote{\url{http://ddosdb.org/}}.

%We apply the to generate signatures and apply this to specific SIDS. We have chosen Bro\footnote{\url{https://www.bro.org/}} as SIDS. Bro is an open source network security monitor that offers the functionality of a SIDS. 



To pursue our goal we have defined the following research questions (RQ) as the basis of the proposed research:

%Intrusion Detection Systems (IDS) is a system that monitors a network or system for malicious and/or suspicious activities within a network or system. Two main cattogories of IDSs are present: \textit{host-based} and \textit{network-based} \cite{Fallahi2016}. A Host-based IDS (HIDS) is only for a single machine. A Network-based IDS (NIDS) is for an entire network. A NIDS usually scans the incoming and outgoing traffic of a network endpoint. Based on the detection methods of IDSs, again two categories can be identified: \textit{Anomaly-based} and \textit{Signature-based} \cite{fragkiadakis2015anomaly}. An Anomaly-based IDS (AIDS) bases its detection on a constructed baseline and detects deviations from this baseline. A Signature-based IDS (SIDS) bases its detection on signatures. An AIDS has as benefit that it can detect unknown attacks but with the weakness that it has a low accuracy and difficulty to trigger alerts in the right time. A SIDS has as benefit that it has a high accuracy but with the weakness that it is ineffective in detecting unknown attacks and it is hard to maintain an up to date signature list \cite{Liao2013}. 



%As the number of attacks are increasing and downtime costs are exceeding on average \$300K per hour \cite{ITIC2016} a need for an efficient and effective mitigation method becomes bigger. We believe that SIDSs are promising systems that can fulfil these requirements when the major downside of keeping and up to date signature list is tackled. That is why we propose an automatic method to generate signatures for the SIDS Bro\footnote{\url{https://www.bro.org/}}. To accomplish this goal we have defined the following research questions (RQ):

\begin{itemize}	
	\item \textbf{RQ1:} What are the current developments in automatic signature generation for SIDSs?
	\item \textbf{RQ2:} What is the performance of automatic signature generation against a DDoS attack for the Bro SIDS?
	 \item \textbf{RQ3:} What is the accuracy and efficiency for Bro automatic generated signatures when applied on an ongoing DDoS attack?
\end{itemize}

The first RQ will be answered by analyzing various IDSs and inspecting various signature generation methods. The second RQ will be answered by building a proof of concept that generates signatures based on a given stream of features of DDoS attacks. The third and last RQ will be answered by replaying an attack for which a signature was generated and analyze what the performance of Bro is with these signatures implemented. 





%\comment{The Introduction section has more or less the same structure as your abstract. The difference is that in the abstract each part is one statement/phrase, while in the introduction each part is a paragraph. So, (i) context, (ii) problem, (iii) proposal, and your most astonishing (iv) finding. Of course in the Introduction section you can give far more details than in the abstract. Avoid to copy and paste statements, re-write with different words.}

%\comment{In addition to the structure that you already know you should include your \textit{research questions} between the ``proposal'' paragraph and the ``findings''. The statement that precede the RQ is something like the following: }

%``To pursue our goal, we have defined the following research questions (RQ) as the basis of our research: 


%\comment{Please, avoid "yes or no" questions. Make questions that your reader are not able to answer immediately. Usually the questions depend on each other, it means that to answer one question you must answer the one before.}

%\comment{Before a little bit of your most astonishing findings you must to introduce the structure of your paper/proposal. Usually the text looks like the following.}
 
The remainder of this proposal is organized as follows. Section 2 discusses the related work identified. In Section \ref{sec:planning} we conclude with a planning for the proposed research. 


