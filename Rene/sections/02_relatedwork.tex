\section{Related Work}
Fallahi et. al. \cite{Fallahi2016} implemented automatic rule generation for the SIDS Snort. They used two data mining algorithms called Ripper and C5.0. They investigated five types of attacks and tested it on the ISCX 2012 dataset. Rather than looking to individual packets, they looked at the flow of data. From the flow 17 features were selected to generate rules. Fallahi et. al. applied this approach to a single dataset which contained various attacks, rather than focusing on DDoS attacks. Furthermore, by looking at the flow of data, the payload is discarded. The payload of messages are for some DDoS attacks a crucial for identification.  

Fouda \cite{Fouda2017} proposes a payload based signature generation specific for DDoS attacks. Various pattern matching algorithms are tested to find the amount of similarity between incoming traffic and traffic that is associated with a DDoS attack. It was found that Smith-Waterman and Longest Common Substring algorithms yielded the highest accuracy. A problem, for example, apparent in this approach is that it tends to become slow. In 2003, a simmilar approach could only cope with up to 200 Mbps of traffic \cite{Lai2004}. Furthermore, not all attacks lend themselves to be accurately identified by their payload. 

Chimetseren et. al \cite{Chimetseren2014} proposes signature based generation method using Discrete Fourier Transform. In this method the payload between client and server are regarded as discrete waveform. They create a spectrum for normal, known attack and unknown attack sessions.  This approach is tested on the Kyoto2006+ dataset\footnote{\url{http://www.takakura.com/Kyoto_data/}}. By also generating a spectrum for normal sessions, they are able to detect unknown attacks. However, this does yield a 5\% false positive in the selected dataset.  

