% You should not remove this section
\section{Conclusion}
\label{section:conclusion}
In this paper, we try to show whether Bro is a suitable SIDS which is capable of detecting attacks with automated generated rules based on the signatures retrieved from DDoSDB. We first show that it is possible to automatically generate Bro signature rules from a JSON file that describes the attack vector's main characteristics. Secondly, we show whether Bro is capable of detecting the attack vectors by executing an experiment that resembles a common internet architecture. In this experiment, we replayed various attack vectors targeting a target machine which port was mirrored to a machine running the Bro IDS. 

From our experiments, we showed that (1) it is possible to generate Bro signatures from various characteristics of attacks. (2) showed the generation time of signatures is reasonable and (3) showed that Bro does not increase dramatically in response time when more signatures are added. However, we do note that there are some limitations to the automatic generation of signatures for Bro such as the byte addressing of the header fields.

For future work, one could investigate the accuracy of the automatically generated rules. One could also further increase the size of the signature list to determine the maximum number of signatures the Bro SIDS can handle. Lastly, it would be interesting to see if the more expressive Bro scripting language can compete with the signature framework in terms of efficiency and accuracy. 

\section{Acknowledgements}
We would like to thank Jair Santanna for his valuable feedback, supplying the tools needed for the test setup and enabling us to gain access to the data stored in DDoSDB. Furthermore, We would like to thank Vincent Dunning for aiding in the process of creating the test setup and carrying out the experiments.  